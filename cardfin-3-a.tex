% cardfin-3-a
% https://docs.google.com/document/d/1LJRhVKVBGR3fm4_ogBEgD7WDbAm-lSzOI02cxd149ME/edit?tab=t.0
% !TEX encoding = UTF-8 Unicode
% https://www.site24x7.com/tools/time-stamp-converter.html 1768130566
\documentclass[12pt,letterpaper]{article}
\usepackage{fancyhdr}
\fancyhf{}
\fancyfoot[R]{{\tiny cardfin-3-a,\ \filemodprintdate{\jobname},\ \filemodprinttime{\jobname},\ 1768130566}} % https://www.site24x7.com/tools/time-stamp-converter.html (https://www.unixtimestamp.com/ bad)
\renewcommand{\headrulewidth}{0pt}
\fancyfoot[C]{\thepage}
\usepackage[T1]{fontenc}
\usepackage[utf8]{inputenc}
\usepackage{amssymb,amsmath,amsthm} 
%\usepackage[letterpaper,top=30pt,left=40pt,right=40pt,bottom=60pt]{geometry}
\usepackage[letterpaper,top=40pt,left=60pt,right=60pt,bottom=70pt]{geometry}
\usepackage{filemod}%https://stackoverflow.com/questions/2118972/latex-command-for-last-modified
\usepackage{microtype}
%\usepackage{tikz-cd}\usepackage{varwidth}\usepackage{comment}
\usepackage[pdfusetitle]{hyperref}
%\usepackage{marvosym}% emoji https://tex.stackexchange.com/questions/3695/smileys-in-latex https://ctan.org/pkg/marvosym?lang=en
\pagestyle{fancy}
%\pagestyle{empty}
%\setlength\parindent{0pt}
\setlength{\parskip}{3pt} % variable
%\renewcommand{\baselinestretch}{1.1} % variable
\newtheorem{thm}{Theorem}%[section]
%\newtheorem{cor}[thm]{Corollary}
\newtheorem{lem}[thm]{Lemma}
\newtheorem{prop}[thm]{Proposition}
\newcommand{\nn}{\noindent}
\newcommand{\N}{\mathbb N}
\begin{document}% \tiny \scriptsize \footnotesize \small \normalsize \large \Large \LARGE \huge \Huge 
\begin{center}
{\large Number of elements of a finite set (version 2)}\footnote{This text is available at \\ \url{https://github.com/Pierre-Yves-Gaillard/Number-of-elements-of-a-finite-set}.}\medskip 

{\footnotesize Pierre-Yves Gaillard}
\end{center}

\nn The statements proved here are almost obvious and not too difficult to find in the literature. I wrote this short text to make them as easily available as possible. 

Let $\N$ be the set of nonnegative integers, and for each $n\in\N$ set $[n]:=\{1,2,\ldots,n\}$ (in particular $[0]=\varnothing$). Let a bijection $[n]\to X$, where $X$ is a set, be called a \textbf{count} of $X$, and let $n$ be called the \textbf{result} of the count. We say that $X$ is \textbf{finite} if it admits a count $[n]\to X$ for some $n\in\N$. 

\begin{thm}\label{T}
If a set $X$ is finite, then any two counts of $X$ give the same result. 
\end{thm} 

If this is the case and the result is $n$, we say that $X$ \textbf{has} $n$ \textbf{elements}. If $X$ and $Y$ are two sets, write $X\simeq Y$ to indicate that there is a bijection $X\to Y$. %The theorem will follow from the proposition below: 

\begin{lem}\label{L2} 
Let $X$ and $Y$ be two sets; let $k$ be in $\N$; let $a_1,\ldots, a_k$ be distinct elements of $X$; let $b_1,\ldots, b_k$ be distinct elements of $Y$; and consider the following conditions: 
\begin{enumerate}
 \item[\emph{(a)}] $X\simeq Y$, 
 \item[\emph{(b)}] $X\setminus\{a_1,\ldots, a_k\}\simeq Y\setminus\{b_1,\ldots, b_k\}$, 
 \item[\emph{(c)}] $X=\{a_1,\ldots, a_k\}\Leftrightarrow Y=\{b_1,\ldots, b_k\}$. 
\end{enumerate} 
Then we have: \emph{(a)} $\Leftrightarrow$ \emph{(b)} $\Rightarrow$ \emph{(c)}.
\end{lem} 

\begin{proof}
We clearly have (a) $\Leftarrow$ (b) $\Rightarrow$ (c). Let us show that (a) $\Rightarrow$ (b). The case $k=0$ is obvious. Assume $k=1$. If $Z$ is any set and $a,b\in Z$, then we define the map $[Z,a,b]:Z\to Z$ by 
$$
[Z,a,b](z)=
\begin{cases}
b&\text{if }z=a\\ 
a&\text{if }z=b\\ 
z&\text{otherwise.}
\end{cases}
$$ 
Then $[Z,a,b]$ is its own inverse; in particular it is bijective. By assumption there is a bijection $f:X\to Y$. Then $[Y,f(a_1),b_1]\circ f$, being a bijection from $X$ to $Y$ mapping $a_1$ to $b_1$, induces a bijection from $X\setminus\{a_1\}$ to $Y\setminus\{b_1\}$. If $k>1$ we have, assuming (as we may) that the implication holds for $k-1$, 
$$
X\setminus\{a_1,\ldots, a_k\}=(X\setminus\{a_1,\ldots, a_{k-1}\})\setminus\{a_k\}\simeq(Y\setminus\{b_1,\ldots, b_{k-1}\})\setminus\{b_k\}= Y\setminus\{b_1,\ldots, b_k\}.
$$ 
\end{proof}

\begin{lem}\label{L3} 
If $[k]\to Z$ is a count of a set $Z$, and if $c_1,\ldots,c_k$ are distinct elements of $Z$, then we have $Z=\{c_1,\ldots,c_k\}$. 
\end{lem} 

\begin{proof}
Use Lemma \ref{L2} with $X=[k],Y=Z$ and $a_i=i,b_i=c_i$ for $i=1,\ldots,k$. 
\end{proof}

\begin{prop}\label{P}
If $m,n\in\N$ satisfy $[m]\simeq[n]$, then we have $m=n$. 
\end{prop} 

\begin{proof}
We assume $m\le n$ (as we may, by symmetry $\simeq$) and apply Lemma \ref{L3}, (a) $\Rightarrow$ (c), with $k=m,Z=[n]$ and $c_i=i$ for $i=1, \ldots, m$. 
\end{proof} 

Theorem \ref{T} now follows immediately from Proposition \ref{P}. 

\end{document}