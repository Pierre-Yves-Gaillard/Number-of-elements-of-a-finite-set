% cardfin-c
% https://docs.google.com/document/d/1Tz553wVIl87WpTgMiK2KIr6x219J3IGTx924iE_6HMo/edit?tab=t.0
% !TEX encoding = UTF-8 Unicode
% https://www.site24x7.com/tools/time-stamp-converter.html 1767041286
\documentclass[12pt,letterpaper]{article}
\usepackage{fancyhdr}
\fancyhf{}
\fancyfoot[R]{{\tiny cardfin-c,\ \filemodprintdate{\jobname},\ \filemodprinttime{\jobname},\ 1767041286}} % https://www.site24x7.com/tools/time-stamp-converter.html (https://www.unixtimestamp.com/ bad)
\renewcommand{\headrulewidth}{0pt}
\fancyfoot[C]{\thepage}
\usepackage[T1]{fontenc}
\usepackage[utf8]{inputenc}
\usepackage{amssymb,amsmath,amsthm} 
%\usepackage[letterpaper,top=30pt,left=40pt,right=40pt,bottom=60pt]{geometry}
\usepackage[letterpaper,top=40pt,left=60pt,right=60pt,bottom=80pt]{geometry}
\usepackage{filemod}%https://stackoverflow.com/questions/2118972/latex-command-for-last-modified
\usepackage{microtype}
%\usepackage{tikz-cd}\usepackage{varwidth}\usepackage{comment}
\usepackage[pdfusetitle]{hyperref}
%\usepackage{marvosym}% emoji https://tex.stackexchange.com/questions/3695/smileys-in-latex https://ctan.org/pkg/marvosym?lang=en
\pagestyle{fancy}
%\pagestyle{empty}
%\setlength\parindent{0pt}
\setlength{\parskip}{3pt} % variable
%\renewcommand{\baselinestretch}{1.1} % variable
\newtheorem{thm}{Theorem}%[section]
%\newtheorem{cor}[thm]{Corollary}
\newtheorem{lem}[thm]{Lemma}
\newtheorem{prop}[thm]{Proposition}
\newcommand{\nn}{\noindent}
\newcommand{\N}{\mathbb N}
\begin{document}% \tiny \scriptsize \footnotesize \small \normalsize \large \Large \LARGE \huge \Huge 
\begin{center}
{\large Number of elements of a finite set}\footnote{This text is available at \\ \url{https://github.com/Pierre-Yves-Gaillard/Number-of-elements-of-a-finite-set}.}\medskip 

{\footnotesize Pierre-Yves Gaillard}
\end{center}

\nn The statements proved here are almost obvious and not too difficult to find in the literature. I wrote this short text to make them as easily available as possible. 

Let $\N$ be the set of nonnegative integers, and, for each $n\in\N$ set $[n]:=\{1,2,\ldots,n\}$ (in particular $[0]=\varnothing$). If $X$ is a set, say that a bijection $[n]\to X$ is a \textbf{count} of $X$, and that $n$ is the \textbf{result} of the count. Say that $X$ is \textbf{finite} if it admits a count $[n]\to X$ for some $n\in\N$. If $X$ and $Y$ are two sets, write $X\simeq Y$ to indicate that there is a bijection $X\to Y$. 

\begin{thm}\label{T}
If a set $X$ is finite, then any two counts of $X$ give the same result. 
\end{thm} 

If such is the case and the result is $n$, we say that $X$ \textbf{has} $n$ \textbf{elements}. The theorem will follow from the proposition below: 

\begin{prop}\label{P}
If $m,n\in\N$ satisfy $[m]\simeq[n]$, then we have $m=n$. 
\end{prop} 

If $X$ is a set and $a,b$ are in $X$, then we define the map $[X,a,b]:X\to X$ by 
$$
[X,a,b](x)=
\begin{cases}
b&\text{if }x=a\\ 
a&\text{if }x=b\\ 
x&\text{otherwise.}
\end{cases}
$$ 
The $[X,a,b]$ is its own inverse; in particular it is bijective. 

\begin{lem}\label{La}
If $X$ and $Y$ are sets such that $X\simeq Y$, and if $a$ is in $X$ and $b$ in $Y$, then there is a bijection $X\to Y$ mapping $a$ to $b$.  
\end{lem} 

\begin{proof}
By assumption there is a bijection $f:X\to Y$. Then $[Y,f(a),b]\circ f$ does the job.  
\end{proof} 

\begin{lem}\label{Lb}
If $X$ and $Y$ are sets such that $X\simeq Y$, and if $a$ is in $X$ and $b$ in $Y$, then we have $X\setminus\{a\}\simeq Y\setminus\{b\}$.  
\end{lem} 

\begin{proof}
By Lemma \ref{La} there is a bijection $f:X\to Y$ mapping $a$ to $b$. Then $f$ induces a bijection $X\setminus\{a\}\to Y\setminus\{b\}$. 
\end{proof} 

\begin{lem}\label{Lc}
If $m$ and $n$ are \emph{positive} integers such that $[m]\simeq[n]$, then we have $[m-1]\simeq[n-1]$.  
\end{lem} 

\begin{proof}
Apply Lemma \ref{Lb} with $X=[m],Y=[n],a=m,b=n$. 
\end{proof} 

\begin{proof}[Proof of Proposition \ref{P}]
Let $m,n\in\N$ satisfy $[m]\simeq[n]$. We must show $m=n$. We can assume $m\le n$. Lemma~\ref{Lc} implies 
$$
[m-1]\simeq[n-1],\ [m-2]\simeq[n-2],\ ...,\ [0]\simeq[n-m],
$$ 
and thus $m=n$, as required. 
\end{proof} 

\end{document}
